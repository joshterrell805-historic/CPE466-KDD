\documentclass{report}

\usepackage{siunitx}
\sisetup{
  round-mode = places,
  round-precision = 3
}
% Format a query and the question the results should answer
\newcommand{\query}[2]{\item\textit{#1}\\#2}

% Format an information need
\newcommand{\infoneed}[2]{
\item \begin{list}{}{}
  \item #1

  \item \itshape#2
  \end{list}
}

% Import a query results file
\newcommand{\results}[2]{\subsection{Query #1 Results}\begin{list}{}{}\input{data/#2}\end{list}}
% Format each result record
\newenvironment{result}[2]{\item #2 \hfill{\small (\num{#1})}\begin{quote}}{\end{quote}}
% Import an info need results file
\newcommand{\needresults}[2]{\subsection{Info Need #1 Results}\begin{list}{}{}\input{data/#2}\end{list}}

\title{CPE466\\Lab2}
\author{
  Gilbert, Andrew\\
  \texttt{apgilber@calpoly.edu}
  \and
  Terrell, Josh\\
  \texttt{jmterrel@calpoly.edu}
}
\date{}

\begin{document}

\maketitle

\begin{abstract}
  When looking at a graph of data, it can be helpful to find the most
influential nodes. The PageRank algorithm provides a straightforward
way to achieve this. We implemented the PageRank algorithm and ran it
on several datasets.
\end{abstract}

\section{Introduction}
The PageRank algorithm, developed by Larry Page and Sergey Brin, and
initially used in what became the Google search engine, is a technique
for calculating the importance of nodes in a graph.



\section{Implementation Overview}
We implemented our reading and writing system in Python, but the bulk
of the algorithm in C. The C runs a master thread and a (configurable)
number of worker threads. Each thread asks the master thread for a
block of data to work on. The data is stored in structs with six
fields:
\begin{lstlisting}[lang=ANSI]
{
  unsigned int id;
  char active;
  double pageRank_a;
  double pageRank_b;
  int outDegree;
  struct LLNode *inNodes;
}
\end{lstlisting}
Each node struct also contains a linked list with pointers to the nodes which link to this one.

Once the graph is built, multiple passes are used to process the
data. On each pass, each worker thread grabs blocks of data and processes them until there is no data left to process.
For each node in the block, the PageRank is read from either the
\texttt{pageRank_a} or \texttt{pageRank_b} field and the new rank is
written to the other. If the rank has changed more than the (configurable) epsilon % Is this correct?
a flag is updated to allow the system to know the algorithm has not
converged. Once all the worker threads have completed their work, if
the ``unconverged'' flag is set, the master thread starts the workers
again, this time telling them to read from the field they were just
writing to and to write to the field they were reading from.

\section{Results}
\subsection{STATES}
\subsection{NCAA-FOOTBALL}
\subsection{KARATE}
\subsection{DOLPHINS}
\subsection{LES-MISERABLES}
\subsection{POLITICAL-BLOGS}
\subsection{WIKI-VOTE}
\subsection{P2P-GNUTELLA05}
\subsection{SLASHDOT-ZOO-NOV6-2008}
\subsection{AMAZON-MAY03}
\subsection{LIVEJOURNAL1}

\section{Overall Summary}
We observered that the NCAA data set is troublesome, because the nodes are in the opposite order from the other datasets. We decided it made more sense to 

\section{Performance Evaluation}
\begin{table}
  \centering
  \begin{tabular}{}
    \toprule
    Dataset & Execution Time\\
    \midrule
    \bottomrule
  \end{tabular}
  \caption{Execution Times on Datasets}
  \label{execution-times-table}
\end{table}
\begin{figure}
  \centering
  
  \caption{Graph of Execution Times on Datasets}
  \label{execution-times-graph}
\end{figure}
\appendix
\section{README}
\lstset{lang=bash}
\textbf{Important: Run everything from the project's root directory}

\subsection{Setup environment}
\begin{lstlisting}
pyvenv virtual
source virtual/bin/activate
pip install --upgrade -e .
\end{lstlisting}

\subsection{Run all the tests}
Run \verb+python3 -m unittest discover+ from the directory containing
this file

\subsection{Calculate and print PageRank for all the data}
\begin{lstlisting}
pagerank data/<dataset>.csv
\end{lstlisting}

\subsection{Deactivate environment}
\begin{lstlisting}
deactivate
\end{lstlisting}
\end{document}

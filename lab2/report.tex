\documentclass{report}

\newcommand{\query}[2]{\item\textit{#1}\\#2}
\title{cpe466-lab2}
\author{jmterrel and apgilber}
\date{October 2015}

\begin{document}

\maketitle

\begin{abstract}
Given some need for information, information retrieval systems are
used to find a relevant subset of documents in a large collection of
documents. To demonstrate and solidify our understanding, we
implemented a simple IR system to query for utterences from the
\textsc{Vaccination-Discussion} dataset. We implemented two matching
algorithms, \textit{cosine similarity} and \textit{okapi}, to build a
result set of relevant documents for a given query.
\end{abstract}

\section{Introduction}
We built a basic IR system to retrieve relevant documents for a given
query. The system includes a component for building vector-space
models for word occurences in documents, and a matching component to
digest these models and ultimatley determine the most relevant
documents to a given query.

\section{System Design and Implementation}
The system consists of two primary components: parsing and query
matching. The parsing component prepares a database of document
vectors and document collection meta data. The query matching
component uses the prepared database for finding the most relevant
documents to the given query.

The parsing component reads in and parses words from the provided JSON
document which serves as the raw document collection. The parsing
phase also does stopword removal and stemming on the document text. A
vector of inner-document word occurrences is appended to each document
and the whole collection is saved to a file. Meta-information
about the collection as a whole, for instance the number of documents
and the document word frequency, is also saved to a file.

The query matching component of the system uses the databases built by
the parsing component to more efficiently find relevant documents to a
given query. It returns the top ten relevant documents, by
default. The query matching component also has command-line flags to
switch between the two implemented matching algorithms, \textit{cosine
similarity} and \textit{okapi}.

\section{Query Answering}
We executed the queries provided using both the \textit{cosine
  similarity} and \textit{okapi} matching algorithms. For each query,
both algorithms were evaluated on their ability to produce relevant
documents in the top ten results (excluding Q4 and Q10, for which we only
examined the first result). The results are displayed in Table 1.

To determine how well the queries produced relevant documents, we
defined an explicit yes/no question per query. The questions we asked
are paired below with each query.

\begin{enumerate}[Q1:]
  \query{Disneyland incident measles people sick}{Is the document
about a Disneyland measles incident?}

  \query{Centers for Disease Control chance of a serious allergic
reaction}{Is the document relevant to both the Centers for Disease
Control and allergic reactions?}

  \query{requiring parents to jump through hoops}{Is the document
about jumping through hoops?}

  \query{Nearly everyone, from my father's generation and generations
older than him, when I'd go and speak to groups, talk about the
stories that they have both from their own childhood, from friends,
personal stories of near-death experiences, friends they had who were
maimed by communicable diseases.}{Does the first result contain the
exact match searched for?}

  \query{parents' right to make healthcare decisions for their
child}{Is the document about parents' rights to make decisions for
their children?}
\end{enumerate}

\begin{table}
  \begin{center}
    \begin{tabu}{c c c}
      \toprule
      Query & Cosine Similarity &  Okapi\\
      \midrule
      1 & 0 & 10\\
      2 & 1 & 2\\
      3 & 2 & 2\\
      4 & y & y\\
      5 & 5 & 8\\
      \bottomrule
    \end{tabu}
  \end{center}
  \caption{Relevant documents in top ten results}
  \label{tab:my_label}
\end{table}

When running the queries through the system, we observed patterns in
the documents that were irrelevant to the queries. One prevailing
observed pattern is that the \textit{cosine similarity} algorithm
tended to produce many large irrelevant documents in the top ten
results. For many of the queries, the same large documents were listed
in the results. On the other hand, the \textit{okapi} algorithm
produced half way relevant documents. For instance, the results for Q2
returned many documents about the Centers for Disease Control or
allergic reaction, but had only two documents relevant to both things.

\section{Information Need Matching}

\section{Analysis and Conclusions}
We initially implemented the basic cosine matching, but we discovered
that it gave too much weight to large documents with many words. We
then implemented Okapi and noticed that our results tended to be much
more relevant.

For both algorithms, we hypothesized that word proximity may also help
in computing more relevant results. For instance, in Q3, having
proximity may give more weight to the words in the phrase ``jump
through hoops'' occurring close together and in the correct order.

Ultimately, we learned some details about IR systems that might have
been much more difficult to learn without the hands-on experience. For
instance, we had no idea that iterating over four-thousand documents
per query would take more than a second, as it did. Using an inverted
index would help dramatically with this performance issue.
\end{document}

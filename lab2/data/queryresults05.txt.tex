\begin{result}{22.431415639841582}{Bonta, Rob (LegStaff) Assembly}
Every child deserves to be able to go to school in a safe and healthy environment. The parents' right to make healthcare decisions for their child is not entirely boundless, and without limitation. It must take into consideration the effect those decisions have on our neighbors, and others in the community.
\end{result}

\begin{result}{22.431415639841582}{Bonta, Rob (Legislator) Assembly}
Every child deserves to be able to go to school in a safe and healthy environment. The parents' right to make healthcare decisions for their child is not entirely boundless, and without limitation. It must take into consideration the effect those decisions have on our neighbors, and others in the community.
\end{result}

\begin{result}{20.616309653426804}{Monning, Bill (Legislator) Senate}
But, where we have a balance here with this bill, that is a tough one for us, is on the one hand, many people voiced their strong belief that they have a right to make the decision for their child. If it were just a decision about their child, I think you would find no quarrel with having a right to make that decision.
\end{result}

\begin{result}{15.940706597810152}{Pan, Richard (Legislator) Assembly}
So remember, this bill is about enrolling children in school. The decision whether to immunize or not is something the parent makes with their doctors or other healthcare providers providing the immunization. That is separate from school entry. When they then bring their child to school, the school will say, please present your immunization record and your school form, and they would say, okay your child may not.
\end{result}

\begin{result}{15.605225435122286}{Loop, Ariel (GeneralPublic) Senate}
There are numerous personal decisions parents make for their children. Disposable vs. cloth diapers, formula vs. breast milk, etc. Vaccination is not a comparable decision. The choice not to vaccinate your children is a threat to public health. Choosing not to vaccinate your child is like, making a choice to risk the health of everyone around you.
\end{result}

\begin{result}{15.03620730889844}{Pan, Richard (Legislator) Assembly}
She asks, why is my child's school made unsafe because of a decision of another parent? What about my child's right to an education? And what of this mother's need to make a living to support her child and family?
\end{result}

\begin{result}{14.527732171017536}{Gordon, Jay (GeneralPublic) Assembly}
And may actually adversely effect children's health. A no vote means you support your constituents rights to make informed medical decisions with their own doctors without coercion. And without the right to free and equal educational access for their children.
\end{result}

\begin{result}{14.1432066442995}{Russin, Leah (GeneralPublic) Senate}
There are parents who disagree with me. I know they want to prevent their children from getting sick. I know some of them personally and I know them to be loving and kind. I respect they are doing what they believe is the best for their child, and if those decisions only affected their own child that might be okay, but their decision not only jeopardizes the health and safety of their children, it puts my child at risk and that is unfair.
\end{result}

\begin{result}{12.773276729732682}{Anderson, Joel (Legislator) Senate}
That here you are not having to take care of that child for the rest of their life, making decisions for them when we know it's a very slim risk but a risk all the same. And you know, I have...I'm not shifting gears because I think that doctor, you've been very insightful.
\end{result}


\begin{result}{24.12091894665227}{Fatheree, George (GeneralPublic) Assembly}
In California we say, kids with exceptional needs. The kids with disabilities must be educated in the least restrictive environment and what that means is to the maximum extent possible, they have a right to receive an education with non-disabled peers and the only exception for that is if the child's disability prevents it. That's the only exception.
\end{result}

\begin{result}{15.144270314277417}{Fatheree, George (GeneralPublic) Assembly}
So there's serious Constitutional problems and under the Federal law, under the Individual with Disabilities Educational Act. One of the pillars of that law, which protects the rights of students with disabilities, is that kids with disabilities have a right to learn with their non-disabled peers.
\end{result}

\begin{result}{14.434302341382798}{Pan, Richard (Legislator) Assembly}
Both protection of the public and children from death and disability due to preventable communicable diseases, and ensuring children with special needs their right to an education, are compelling interests to require all students be vaccinated unless granted a medical exemption.
\end{result}

\begin{result}{13.880531196968432}{Pan, Richard (Legislator) Senate}
Both protection of the public and children from death and disability due to preventable communicable diseases and assuring that children with special needs, their right to an education, are compelling interests to require all students be vaccinated unless they are granted a medical exemption. Thank you.
\end{result}

\begin{result}{13.081970355015326}{Fatheree, George (GeneralPublic) Assembly}
The second part of the test is the law has to be narrowly tailored to limit rights, the right to education as little as possible, which I think this bill is the opposite. It's very broad. That's just the constitutional analysis under California's constitution. In addition, we've got the Individuals with Disabilities Educational Improvement Act, the IDEA, which is a federal law. California mirrors federal law in this respect, that requires that kids with special needs.
\end{result}

\begin{result}{12.771668141654398}{Holland, Mary (GeneralPublic) Senate}
I also want to touch briefly on the legal problems that would arise in taking away education from special needs children under the Individuals with Disability Education Act. Under federal law, special needs children are entitled to a free and public education that includes annual individualized educational plan reviews and a host of special services, including occupational therapy, physical therapy, and speech therapy.
\end{result}

\begin{result}{11.435643821597985}{Bonta, Rob (LegStaff) Assembly}
Thank you. The next is with respect to individualized education plans and special education. That is, jumping down now to section H. And as discussed in the analysis under Federal and state law, disabled children are guaranteed the right to a free appropriate public education, including necessary services for a child to benefit from his or her education.
\end{result}

\begin{result}{11.435643821597985}{Bonta, Rob (Legislator) Assembly}
Thank you. The next is with respect to individualized education plans and special education. That is, jumping down now to section H. And as discussed in the analysis under Federal and state law, disabled children are guaranteed the right to a free appropriate public education, including necessary services for a child to benefit from his or her education.
\end{result}

\begin{result}{10.608660701613644}{Amador Nava, Shelby (GeneralPublic) Senate}
Shelby Amador Nava, I'm an educator, child care provider for special needs kids and a single mother and I strongly oppose this bill.
\end{result}


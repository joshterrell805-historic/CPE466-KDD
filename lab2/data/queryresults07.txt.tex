\begin{result}{22.398952927698552}{Pan, Richard (Legislator) Senate}
\textbf{So the exemption that is in statute basically says that any licensed physician, any licensed physician in the state of California can grant a medical exemption.
}\end{result}

\begin{result}{21.591685490649436}{Pan, Richard (Legislator) Assembly}
\textbf{The law clearly states that any licensed physician in the state of California can provide a medical exemption. Mainly what they have to do is document the reason. They document the duration, which there's no limitation. That can be indefinite. And they have to sign it, of course, saying that they are a licensed physician. And you get a medical exemption.
}\end{result}

\begin{result}{19.281240261790572}{Pan, Richard (Legislator) Senate}
\textbf{So, actually, existing law for the medical exemption allows a licensed physician in the State of California to exempt a child with a medical exemption for vaccination. And so, if the licensed physician believes that that child is at risk for injury from the vaccine, they can simply give a medical exemption. In many ways the law has no restrictions on a licensed physician in being able to do that.
}\end{result}

\begin{result}{18.572197801528077}{Pan, Richard (Legislator) Senate}
\textbf{that their licensed physician would not write a medical exemption, again any licensed physician can write that medical exemption.
}\end{result}

\begin{result}{17.785507332229226}{Pan, Richard (Legislator) Assembly}
\textbf{That is, it's the professional challenge of the physician in terms of what they believe, that if the risk of the immunization is gonna be such that it's going to put that child at certain or, near, basically increased harm. Then they can provide that exemption. And so, that's to the judgment of actually any licensed physician in the state of California. There's no requirement that you even have to go to a physician that you've seen multiple times in the past.
}\end{result}

\begin{result}{15.941329185646078}{Pan, Richard (Legislator) Senate}
They certainly are, any licensed physician in the state of California.
\end{result}

\begin{result}{15.542348794795622}{Reiss, Dorit (GeneralPublic) Senate}
In the past, California was extremely generous in providing exemptions from immunization requirements. Before AB 2109 it was even easier than it is now. It still is reasonably easy to get an exemption, but again this balance is something states can and do consider as things change.
\end{result}

\begin{result}{14.421991099835951}{Pan, Richard (Legislator) Senate}
\textbf{So if you're under the care of one, 2 specialists, primary care there's no restriction any licensed physician can write a medical exemption per the law.
}\end{result}

\begin{result}{13.499722256772912}{Pan, Richard (Legislator) Senate}
\textbf{There is a definition in statute. We did not touch that in the bill which is why you don't see it in the bill language. But it basically says that a licensed physician in the state of California and their professional judgement can grant the medical exception.
}\end{result}

\begin{result}{13.032809048933341}{Pan, Richard (Legislator) Senate}
\textbf{There is no other restriction on the judgement of a licensed physician, so anyone who said that they have a licensed physician who told them they couldn't get immunized, should be able to get a medical exemption, and then those children, because they're not immunized, require protection. When you talk about do you have children who may be partially immunized, they may have certain vaccines, they may not take other vaccines.
}\end{result}

